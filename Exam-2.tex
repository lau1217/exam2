% Options for packages loaded elsewhere
\PassOptionsToPackage{unicode}{hyperref}
\PassOptionsToPackage{hyphens}{url}
%
\documentclass[
]{article}
\usepackage{amsmath,amssymb}
\usepackage{lmodern}
\usepackage{ifxetex,ifluatex}
\ifnum 0\ifxetex 1\fi\ifluatex 1\fi=0 % if pdftex
  \usepackage[T1]{fontenc}
  \usepackage[utf8]{inputenc}
  \usepackage{textcomp} % provide euro and other symbols
\else % if luatex or xetex
  \usepackage{unicode-math}
  \defaultfontfeatures{Scale=MatchLowercase}
  \defaultfontfeatures[\rmfamily]{Ligatures=TeX,Scale=1}
\fi
% Use upquote if available, for straight quotes in verbatim environments
\IfFileExists{upquote.sty}{\usepackage{upquote}}{}
\IfFileExists{microtype.sty}{% use microtype if available
  \usepackage[]{microtype}
  \UseMicrotypeSet[protrusion]{basicmath} % disable protrusion for tt fonts
}{}
\makeatletter
\@ifundefined{KOMAClassName}{% if non-KOMA class
  \IfFileExists{parskip.sty}{%
    \usepackage{parskip}
  }{% else
    \setlength{\parindent}{0pt}
    \setlength{\parskip}{6pt plus 2pt minus 1pt}}
}{% if KOMA class
  \KOMAoptions{parskip=half}}
\makeatother
\usepackage{xcolor}
\IfFileExists{xurl.sty}{\usepackage{xurl}}{} % add URL line breaks if available
\IfFileExists{bookmark.sty}{\usepackage{bookmark}}{\usepackage{hyperref}}
\hypersetup{
  pdftitle={Exam 2},
  hidelinks,
  pdfcreator={LaTeX via pandoc}}
\urlstyle{same} % disable monospaced font for URLs
\usepackage[margin=1in]{geometry}
\usepackage{color}
\usepackage{fancyvrb}
\newcommand{\VerbBar}{|}
\newcommand{\VERB}{\Verb[commandchars=\\\{\}]}
\DefineVerbatimEnvironment{Highlighting}{Verbatim}{commandchars=\\\{\}}
% Add ',fontsize=\small' for more characters per line
\usepackage{framed}
\definecolor{shadecolor}{RGB}{248,248,248}
\newenvironment{Shaded}{\begin{snugshade}}{\end{snugshade}}
\newcommand{\AlertTok}[1]{\textcolor[rgb]{0.94,0.16,0.16}{#1}}
\newcommand{\AnnotationTok}[1]{\textcolor[rgb]{0.56,0.35,0.01}{\textbf{\textit{#1}}}}
\newcommand{\AttributeTok}[1]{\textcolor[rgb]{0.77,0.63,0.00}{#1}}
\newcommand{\BaseNTok}[1]{\textcolor[rgb]{0.00,0.00,0.81}{#1}}
\newcommand{\BuiltInTok}[1]{#1}
\newcommand{\CharTok}[1]{\textcolor[rgb]{0.31,0.60,0.02}{#1}}
\newcommand{\CommentTok}[1]{\textcolor[rgb]{0.56,0.35,0.01}{\textit{#1}}}
\newcommand{\CommentVarTok}[1]{\textcolor[rgb]{0.56,0.35,0.01}{\textbf{\textit{#1}}}}
\newcommand{\ConstantTok}[1]{\textcolor[rgb]{0.00,0.00,0.00}{#1}}
\newcommand{\ControlFlowTok}[1]{\textcolor[rgb]{0.13,0.29,0.53}{\textbf{#1}}}
\newcommand{\DataTypeTok}[1]{\textcolor[rgb]{0.13,0.29,0.53}{#1}}
\newcommand{\DecValTok}[1]{\textcolor[rgb]{0.00,0.00,0.81}{#1}}
\newcommand{\DocumentationTok}[1]{\textcolor[rgb]{0.56,0.35,0.01}{\textbf{\textit{#1}}}}
\newcommand{\ErrorTok}[1]{\textcolor[rgb]{0.64,0.00,0.00}{\textbf{#1}}}
\newcommand{\ExtensionTok}[1]{#1}
\newcommand{\FloatTok}[1]{\textcolor[rgb]{0.00,0.00,0.81}{#1}}
\newcommand{\FunctionTok}[1]{\textcolor[rgb]{0.00,0.00,0.00}{#1}}
\newcommand{\ImportTok}[1]{#1}
\newcommand{\InformationTok}[1]{\textcolor[rgb]{0.56,0.35,0.01}{\textbf{\textit{#1}}}}
\newcommand{\KeywordTok}[1]{\textcolor[rgb]{0.13,0.29,0.53}{\textbf{#1}}}
\newcommand{\NormalTok}[1]{#1}
\newcommand{\OperatorTok}[1]{\textcolor[rgb]{0.81,0.36,0.00}{\textbf{#1}}}
\newcommand{\OtherTok}[1]{\textcolor[rgb]{0.56,0.35,0.01}{#1}}
\newcommand{\PreprocessorTok}[1]{\textcolor[rgb]{0.56,0.35,0.01}{\textit{#1}}}
\newcommand{\RegionMarkerTok}[1]{#1}
\newcommand{\SpecialCharTok}[1]{\textcolor[rgb]{0.00,0.00,0.00}{#1}}
\newcommand{\SpecialStringTok}[1]{\textcolor[rgb]{0.31,0.60,0.02}{#1}}
\newcommand{\StringTok}[1]{\textcolor[rgb]{0.31,0.60,0.02}{#1}}
\newcommand{\VariableTok}[1]{\textcolor[rgb]{0.00,0.00,0.00}{#1}}
\newcommand{\VerbatimStringTok}[1]{\textcolor[rgb]{0.31,0.60,0.02}{#1}}
\newcommand{\WarningTok}[1]{\textcolor[rgb]{0.56,0.35,0.01}{\textbf{\textit{#1}}}}
\usepackage{graphicx}
\makeatletter
\def\maxwidth{\ifdim\Gin@nat@width>\linewidth\linewidth\else\Gin@nat@width\fi}
\def\maxheight{\ifdim\Gin@nat@height>\textheight\textheight\else\Gin@nat@height\fi}
\makeatother
% Scale images if necessary, so that they will not overflow the page
% margins by default, and it is still possible to overwrite the defaults
% using explicit options in \includegraphics[width, height, ...]{}
\setkeys{Gin}{width=\maxwidth,height=\maxheight,keepaspectratio}
% Set default figure placement to htbp
\makeatletter
\def\fps@figure{htbp}
\makeatother
\setlength{\emergencystretch}{3em} % prevent overfull lines
\providecommand{\tightlist}{%
  \setlength{\itemsep}{0pt}\setlength{\parskip}{0pt}}
\setcounter{secnumdepth}{-\maxdimen} % remove section numbering
\ifluatex
  \usepackage{selnolig}  % disable illegal ligatures
\fi

\title{Exam 2}
\author{}
\date{\vspace{-2.5em}}

\begin{document}
\maketitle

\#Exam 2 - Data Science for the Social World \#\#Laura Morales
\#\#\#June 28, 2021 \textbf{1. Please clear the environment in R.}

\begin{Shaded}
\begin{Highlighting}[]
\FunctionTok{rm}\NormalTok{(}\AttributeTok{list=}\FunctionTok{ls}\NormalTok{(}\AttributeTok{all=}\ConstantTok{TRUE}\NormalTok{))}
\end{Highlighting}
\end{Shaded}

\textbf{2. Load the college\_scorecard dataset in R, and call it
``college\_scorecard''.}

\begin{Shaded}
\begin{Highlighting}[]
\FunctionTok{library}\NormalTok{(rio) }\CommentTok{\#rio package to load data}
\NormalTok{college\_scorecard }\OtherTok{=} \FunctionTok{import}\NormalTok{(}\StringTok{"2021\_exam2\_data.xlsx"}\NormalTok{, }\AttributeTok{which=}\DecValTok{4}\NormalTok{) }\CommentTok{\#load data}
\end{Highlighting}
\end{Shaded}

\begin{enumerate}
\def\labelenumi{\arabic{enumi}.}
\setcounter{enumi}{2}
\tightlist
\item
  Provide summary statistics for the college\_scorecard dataset.
\end{enumerate}

\begin{Shaded}
\begin{Highlighting}[]
\FunctionTok{summary}\NormalTok{(college\_scorecard)}
\end{Highlighting}
\end{Shaded}

\begin{verbatim}
##      unitid        inst_name          state_abbr       
##  Min.   :100654   Length:48445       Length:48445      
##  1st Qu.:163532   Class :character   Class :character  
##  Median :212115   Mode  :character   Mode  :character  
##  Mean   :260438                                        
##  3rd Qu.:409120                                        
##  Max.   :490009                                        
##                                                        
##  pred_degree_awarded_ipeds      year       earnings_med    count_not_working
##  Min.   :1.000             Min.   :2007   Min.   :  8400   Min.   :    0.0  
##  1st Qu.:1.000             1st Qu.:2011   1st Qu.: 24700   1st Qu.:   46.0  
##  Median :2.000             Median :2012   Median : 31600   Median :  115.0  
##  Mean   :1.913             Mean   :2012   Mean   : 33348   Mean   :  369.4  
##  3rd Qu.:3.000             3rd Qu.:2014   3rd Qu.: 39800   3rd Qu.:  300.0  
##  Max.   :3.000             Max.   :2016   Max.   :186500   Max.   :15960.0  
##                                           NA's   :15706    NA's   :15801    
##  count_working  
##  Min.   :    8  
##  1st Qu.:  210  
##  Median :  594  
##  Mean   : 2073  
##  3rd Qu.: 1477  
##  Max.   :94724  
##  NA's   :14772
\end{verbatim}

\textbf{4. students who graduated from four-year+ colleges and
universities located in Texas (state\_abbr: ``TX'') and Louisiana
(state\_abbr: ``LA''). Call the resulting data frame
``small\_scorecard''.}

\begin{Shaded}
\begin{Highlighting}[]
\NormalTok{small\_scorecard }\OtherTok{=} \FunctionTok{subset}\NormalTok{(college\_scorecard, college\_scorecard}\SpecialCharTok{$}\NormalTok{state\_abbr }\SpecialCharTok{==} \StringTok{\textquotesingle{}TX\textquotesingle{}} \SpecialCharTok{|}\NormalTok{ college\_scorecard}\SpecialCharTok{$}\NormalTok{state\_abbr }\SpecialCharTok{==} \StringTok{\textquotesingle{}LA\textquotesingle{}}\NormalTok{) }\CommentTok{\#subset the colleges based on the states abbreviation}
\NormalTok{small\_scorecard }\OtherTok{=} \FunctionTok{subset}\NormalTok{(small\_scorecard, small\_scorecard}\SpecialCharTok{$}\NormalTok{year }\SpecialCharTok{\textgreater{}=} \DecValTok{2014}\NormalTok{)}\CommentTok{\#subset the colleges based on the lower limit of the year}
\NormalTok{small\_scorecard }\OtherTok{=} \FunctionTok{subset}\NormalTok{(small\_scorecard, small\_scorecard}\SpecialCharTok{$}\NormalTok{year }\SpecialCharTok{\textless{}} \DecValTok{2016}\NormalTok{)}\CommentTok{\#subset the colleges based on the upper limit of the year}
\end{Highlighting}
\end{Shaded}

\textbf{5. Collapse the ``small\_scorecard'' data frame to get both (a)
the average of number people working who graduated from universities in
Texas and Lousiana; and (b) the total number of people working who
graduated from universities in Texas and Lousiana. Call your resulting
data frame ``even\_smaller\_scorecard''. (Hint: Your resulting data
frame should only have two observations--one for Texas, the other for
Lousiana.)}

\begin{Shaded}
\begin{Highlighting}[]
\CommentTok{\#load the dplyr library}
\FunctionTok{library}\NormalTok{(dplyr)}
\end{Highlighting}
\end{Shaded}

\begin{verbatim}
## 
## Attaching package: 'dplyr'
\end{verbatim}

\begin{verbatim}
## The following objects are masked from 'package:stats':
## 
##     filter, lag
\end{verbatim}

\begin{verbatim}
## The following objects are masked from 'package:base':
## 
##     intersect, setdiff, setequal, union
\end{verbatim}

\begin{Shaded}
\begin{Highlighting}[]
\CommentTok{\#pipe the small score card}
\NormalTok{even\_smaller\_scorecard }\OtherTok{\textless{}{-}}\NormalTok{  small\_scorecard }\SpecialCharTok{\%\textgreater{}\%} 
  \FunctionTok{na.omit}\NormalTok{(small\_scorecard, }\AttributeTok{select=} \FunctionTok{c}\NormalTok{(}\StringTok{"count\_not\_working"}\NormalTok{, }\StringTok{"count\_working"}\NormalTok{)) }\SpecialCharTok{\%\textgreater{}\%} \CommentTok{\#drop the na values}
  \FunctionTok{group\_by}\NormalTok{(state\_abbr) }\SpecialCharTok{\%\textgreater{}\%} \CommentTok{\#group the states }
  \FunctionTok{summarize}\NormalTok{(}\FunctionTok{across}\NormalTok{(}\FunctionTok{where}\NormalTok{(is.numeric), sum)) }\CommentTok{\#summarize count the amount of degrees}
\end{Highlighting}
\end{Shaded}

\textbf{6. Use the ``even\_smaller\_scorecard'' data frame to provide a
bar graph detail- ing the percent of people working. Make sure to label
the axes and provide an appropriate title for the graph. (Hint: you will
need to create a new variable to answer this question.)}

\begin{Shaded}
\begin{Highlighting}[]
\NormalTok{even\_smaller\_scorecard}\SpecialCharTok{$}\NormalTok{percent\_working }\OtherTok{\textless{}{-}}\NormalTok{ (even\_smaller\_scorecard}\SpecialCharTok{$}\NormalTok{count\_not\_working }\SpecialCharTok{/}\NormalTok{ (even\_smaller\_scorecard}\SpecialCharTok{$}\NormalTok{count\_not\_working}\SpecialCharTok{+}\NormalTok{even\_smaller\_scorecard}\SpecialCharTok{$}\NormalTok{count\_working))}\SpecialCharTok{*}\DecValTok{100}
\FunctionTok{library}\NormalTok{(ggplot2)}
\NormalTok{barplot }\OtherTok{=} \FunctionTok{ggplot}\NormalTok{(}\AttributeTok{data=}\NormalTok{ even\_smaller\_scorecard, }\FunctionTok{aes}\NormalTok{(even\_smaller\_scorecard}\SpecialCharTok{$}\NormalTok{state\_abbr, even\_smaller\_scorecard}\SpecialCharTok{$}\NormalTok{percent\_working))}\SpecialCharTok{+}
  \FunctionTok{geom\_bar}\NormalTok{(}\AttributeTok{stat=}\StringTok{"identity"}\NormalTok{)}\SpecialCharTok{+}
  \FunctionTok{labs}\NormalTok{(}\AttributeTok{x=} \StringTok{"State"}\NormalTok{, }
       \AttributeTok{y=} \StringTok{"Percent of people working"}\NormalTok{,}
       \AttributeTok{title=} \StringTok{"Percent of students working in the state"}\NormalTok{)}
\FunctionTok{print}\NormalTok{(barplot)}
\end{Highlighting}
\end{Shaded}

\begin{verbatim}
## Warning: Use of `even_smaller_scorecard$state_abbr` is discouraged. Use
## `state_abbr` instead.
\end{verbatim}

\begin{verbatim}
## Warning: Use of `even_smaller_scorecard$percent_working` is discouraged. Use
## `percent_working` instead.
\end{verbatim}

\includegraphics{Exam-2_files/figure-latex/unnamed-chunk-6-1.pdf}
\textbf{7. On the basis of your graph, did people who graduated from
four-year colleges/universities located in Texas or Louisiana have a
better chance of being employed? More broadly, do you think that going
to college/university in one state gives people a better chance at
getting a job? (Hints: (a) you will want to take a look at the summary
statistics of the ``even\_smaller\_scorecard'' data frame; and (b) you
will want to take a look at the universities included in the
``smaller\_scorecard'')} The people who graduated from colleges and
universities located in Texas have a better chance of being employed.

\begin{Shaded}
\begin{Highlighting}[]
\FunctionTok{summary}\NormalTok{(even\_smaller\_scorecard) }\CommentTok{\#summary of the two states}
\end{Highlighting}
\end{Shaded}

\begin{verbatim}
##   state_abbr            unitid          pred_degree_awarded_ipeds
##  Length:2           Min.   : 23369594   Min.   :144.0            
##  Class :character   1st Qu.: 45647498   1st Qu.:259.8            
##  Mode  :character   Median : 67925402   Median :375.5            
##                     Mean   : 67925402   Mean   :375.5            
##                     3rd Qu.: 90203307   3rd Qu.:491.2            
##                     Max.   :112481211   Max.   :607.0            
##       year         earnings_med      count_not_working count_working   
##  Min.   :167162   Min.   : 2499300   Min.   : 36477    Min.   :205786  
##  1st Qu.:298072   1st Qu.: 4573125   1st Qu.: 63268    1st Qu.:343627  
##  Median :428982   Median : 6646950   Median : 90058    Median :481468  
##  Mean   :428982   Mean   : 6646950   Mean   : 90058    Mean   :481468  
##  3rd Qu.:559892   3rd Qu.: 8720775   3rd Qu.:116848    3rd Qu.:619310  
##  Max.   :690802   Max.   :10794600   Max.   :143639    Max.   :757151  
##  percent_working
##  Min.   :15.06  
##  1st Qu.:15.28  
##  Median :15.50  
##  Mean   :15.50  
##  3rd Qu.:15.72  
##  Max.   :15.95
\end{verbatim}

\textbf{8. }

\begin{Shaded}
\begin{Highlighting}[]
\FunctionTok{library}\NormalTok{(rio)}
\NormalTok{avocados }\OtherTok{=} \FunctionTok{import}\NormalTok{(}\StringTok{"2021\_exam2\_data.xlsx"}\NormalTok{, }\AttributeTok{which=}\DecValTok{2}\NormalTok{)}
\end{Highlighting}
\end{Shaded}

\textbf{9. }

\begin{Shaded}
\begin{Highlighting}[]
\FunctionTok{library}\NormalTok{(tidyverse)}
\end{Highlighting}
\end{Shaded}

\begin{verbatim}
## -- Attaching packages --------------------------------------- tidyverse 1.3.1 --
\end{verbatim}

\begin{verbatim}
## v tibble  3.1.2     v purrr   0.3.4
## v tidyr   1.1.3     v stringr 1.4.0
## v readr   1.4.0     v forcats 0.5.1
\end{verbatim}

\begin{verbatim}
## -- Conflicts ------------------------------------------ tidyverse_conflicts() --
## x dplyr::filter() masks stats::filter()
## x dplyr::lag()    masks stats::lag()
\end{verbatim}

\begin{Shaded}
\begin{Highlighting}[]
\FunctionTok{library}\NormalTok{(lubridate)}
\end{Highlighting}
\end{Shaded}

\begin{verbatim}
## 
## Attaching package: 'lubridate'
\end{verbatim}

\begin{verbatim}
## The following objects are masked from 'package:base':
## 
##     date, intersect, setdiff, union
\end{verbatim}

\begin{Shaded}
\begin{Highlighting}[]
\NormalTok{avocados }\OtherTok{\textless{}{-}}\NormalTok{ avocados }\SpecialCharTok{\%\textgreater{}\%}
\NormalTok{dplyr}\SpecialCharTok{::}\FunctionTok{mutate}\NormalTok{(}\AttributeTok{year =}\NormalTok{ lubridate}\SpecialCharTok{::}\FunctionTok{year}\NormalTok{(avocados}\SpecialCharTok{$}\NormalTok{date))}
\end{Highlighting}
\end{Shaded}

\textbf{10. }

\begin{Shaded}
\begin{Highlighting}[]
\CommentTok{\#add World development indicators (WDI)}
\FunctionTok{library}\NormalTok{(WDI) }\CommentTok{\#load WDI package}
\NormalTok{deflator }\OtherTok{=} \FunctionTok{WDI}\NormalTok{(}\AttributeTok{country =} \StringTok{"US"}\NormalTok{, }\AttributeTok{indicator =} \FunctionTok{c}\NormalTok{(}\StringTok{"NY.GDP.DEFL.ZS"}\NormalTok{), }\AttributeTok{start =} \DecValTok{1960}\NormalTok{, }\AttributeTok{end =} \DecValTok{2018}\NormalTok{, }\AttributeTok{extra =} \ConstantTok{FALSE}\NormalTok{, }\AttributeTok{cache =} \ConstantTok{NULL}\NormalTok{) }\CommentTok{\#create variable from the package}
\FunctionTok{library}\NormalTok{(data.table) }\CommentTok{\#load data table package}
\end{Highlighting}
\end{Shaded}

\begin{verbatim}
## 
## Attaching package: 'data.table'
\end{verbatim}

\begin{verbatim}
## The following objects are masked from 'package:lubridate':
## 
##     hour, isoweek, mday, minute, month, quarter, second, wday, week,
##     yday, year
\end{verbatim}

\begin{verbatim}
## The following object is masked from 'package:purrr':
## 
##     transpose
\end{verbatim}

\begin{verbatim}
## The following objects are masked from 'package:dplyr':
## 
##     between, first, last
\end{verbatim}

\begin{Shaded}
\begin{Highlighting}[]
\FunctionTok{setnames}\NormalTok{(deflator, }\StringTok{"NY.GDP.DEFL.ZS"}\NormalTok{, }\StringTok{"deflator"}\NormalTok{) }\CommentTok{\#rename}
\NormalTok{deflated\_data }\OtherTok{=} \FunctionTok{left\_join}\NormalTok{(}\AttributeTok{x=}\NormalTok{avocados,}
                          \AttributeTok{y=}\NormalTok{deflator,}
                          \AttributeTok{by=}\StringTok{\textquotesingle{}year\textquotesingle{}}\NormalTok{) }\CommentTok{\#left join the deflator and avocados dataframs}
\NormalTok{deflated\_data}\SpecialCharTok{$}\NormalTok{deflated\_amount }\OtherTok{=}\NormalTok{ deflated\_data}\SpecialCharTok{$}\NormalTok{average\_price}\SpecialCharTok{/}\NormalTok{(deflated\_data}\SpecialCharTok{$}\NormalTok{deflator}\SpecialCharTok{/}\DecValTok{100}\NormalTok{) }\CommentTok{\#adjust for inflation}
\end{Highlighting}
\end{Shaded}

\textbf{11. }

\begin{Shaded}
\begin{Highlighting}[]
\NormalTok{collapsed\_avocados }\OtherTok{\textless{}{-}}\NormalTok{ deflated\_data }\SpecialCharTok{\%\textgreater{}\%} 
  \FunctionTok{group\_by}\NormalTok{(year) }\SpecialCharTok{\%\textgreater{}\%} \CommentTok{\#group the year }
  \FunctionTok{summarize}\NormalTok{(}\FunctionTok{across}\NormalTok{(}\FunctionTok{where}\NormalTok{(is.numeric), mean)) }\CommentTok{\#summarize mean price}
\end{Highlighting}
\end{Shaded}

\textbf{12. }

\begin{Shaded}
\begin{Highlighting}[]
\NormalTok{wide\_avocados }\OtherTok{\textless{}{-}}\NormalTok{ collapsed\_avocados }\SpecialCharTok{\%\textgreater{}\%} \FunctionTok{pivot\_wider}\NormalTok{(}\AttributeTok{id\_cols =} \FunctionTok{c}\NormalTok{ ( }\StringTok{"total\_volume"}\NormalTok{, }\StringTok{"average\_price"}\NormalTok{, }\StringTok{"year"}\NormalTok{), }\AttributeTok{names\_from =} \StringTok{"year"}\NormalTok{, }\AttributeTok{values\_from =} \StringTok{"deflated\_amount"}\NormalTok{ )}\CommentTok{\#omake the year the columns}
\FunctionTok{head}\NormalTok{(wide\_avocados)}\CommentTok{\#only see the first five values}
\end{Highlighting}
\end{Shaded}

\begin{verbatim}
## # A tibble: 4 x 6
##   total_volume average_price `2015` `2016` `2017` `2018`
##          <dbl>         <dbl>  <dbl>  <dbl>  <dbl>  <dbl>
## 1     5681498.          1.02   1.02  NA     NA     NA   
## 2     6105539.          1.05  NA      1.04  NA     NA   
## 3     5834479.          1.25  NA     NA      1.22  NA   
## 4     6786962.          1.08  NA     NA     NA      1.02
\end{verbatim}

\textbf{13. }

\begin{Shaded}
\begin{Highlighting}[]
\FunctionTok{library}\NormalTok{(data.table)}\CommentTok{\#load library}
\FunctionTok{setnames}\NormalTok{(wide\_avocados, }\StringTok{"total\_volume"}\NormalTok{, }\StringTok{"avocado\_total\_volume"}\NormalTok{)}\CommentTok{\#set names to be more descriptive }
\end{Highlighting}
\end{Shaded}

\textbf{14. }

\begin{Shaded}
\begin{Highlighting}[]
\NormalTok{college\_scorecard }\OtherTok{=} \FunctionTok{import}\NormalTok{(}\StringTok{"2021\_exam2\_data.xlsx"}\NormalTok{, }\AttributeTok{which=}\DecValTok{4}\NormalTok{) }\CommentTok{\#load data}
\end{Highlighting}
\end{Shaded}

\textbf{15. }

\textbf{16. }

\end{document}
